% This is samplepaper.tex, a sample chapter demonstrating the
% LLNCS macro package for Springer Computer Science proceedings;
% Version 2.20 of 2017/10/04
%
\documentclass[runningheads]{llncs}
%
\usepackage{xurl}
\usepackage{graphicx}
\usepackage{textcomp}
\usepackage[portuguese]{babel}
\usepackage{multirow}
\usepackage{imakeidx}
\makeindex
\usepackage{titlesec}


% Used for displaying a sample figure. If possible, figure files should
% be included in EPS format.
%
% If you use the hyperref package, please uncomment the following line
% to display URLs in blue roman font according to Springer's eBook style:
% \renewcommand\UrlFont{\color{blue}\rmfamily}


\title{Trabalho Prático Nº1 – Protocolo de Camada de Transporte}
\author{Carlos Gustavo Silva Pereira a96867, Cláudio Alexandre Freitas Bessa a97063, João Miguel Ferreira Loureiro a97257}
\institute{Universidade do Minho - Licenciatura em Engenharia Informática}
\authorrunning{Gustavo Pereira \and Cláudio Bessa \and João Loureiro}

\begin{document}
\maketitle

\paragraph{}
\paragraph{}
\paragraph{}
\paragraph{}
\paragraph{}
\paragraph{}
\paragraph{}
\paragraph{}
\paragraph{}
\paragraph{}
\paragraph{}
\paragraph{}
\paragraph{}
\paragraph{}
\paragraph{}

\begin{figure}[ht]
\centering
\includegraphics[scale=0.15]{img/na_universidade_do_minho-1.png} 
\end{figure} 

\setcounter{tocdepth}{2}
\makeatletter
\renewcommand*\l@author[2]{}
\renewcommand*\l@title[2]{}
\makeatletter
        
\newpage
\renewcommand{\contentsname}{Índice}
\tableofcontents
\printindex

\pagebreak

%
\renewcommand{\abstractname}{\textbf{Resumo:}}
\renewcommand{\keywords}{\textbf{Palavras-chave: }}
\begin{abstract}
O estudo dos protocolo que compõe a camada de transporte é um dos pilares da comunicação por computadores. Este documento aborda noções importantes relativas ao transporte de dados numa rede demonstrando a relevância dos protocolos TCP e UDP em diferentes contextos.
Para este trabalho utilizamos auxiliamo-nos de uma maquina virtual com o sistema operativo \textit{Xubuntu} e os \textit{softwares} \textit{CORE}, para emulação de uma rede, e o \textit{Wireshark}, para visualização das trocas de pacotes.

\keywords{Protocolo de Camada de Transporte,  Comunicação por Computadores, Segmentos, Pacotes e Datagrama}
\end{abstract}

\section{Questões}
    \subsection{Pergunta 1}
    De que forma as perdas e duplicações de pacotes afetaram o desempenho das aplicações? Que camada lidou com esses problemas: transporte ou aplicação? Responda com base nas experiências feitas e nos resultados observados.
    
    \bigskip
    
    \textbf{Resposta}: Após a verificação de ambos os \textit{pings} nos diferentes ambientes da topologia, verificamos que o computador \textit{Grilo} foi mais lento, uma vez que essa rede tinha associada uma determinada probabilidade de perda e duplicações. 
    A camada que lida com estes tipos de problemas é a camada da aplicação.
    
    \begin{figure}[!ht]
    \centering
    \includegraphics[scale=0.5]{img/a1 1.png}
    \caption{Resultados obtidos em Portatil1} 
    \label{fig:data1}
    \end{figure}
    
    \begin{figure}[!ht]
    \centering
    \includegraphics[scale=0.5]{img/a1 2.png}
    \caption{Resultados obtidos em Grilo} 
    \label{fig:data2}
    \end{figure}
    
    \newpage
    
    \subsection{Pergunta 2}
    Obtenha a partir do \textit{Wireshark}, ou desenhe manualmente, um diagrama temporal para a transferência do ficheiro \textit{file1} por \textit{FTP} realizada em A.3. Foque-se apenas na transferência de dados [\textit{ftp-data}] e não na conexão de controlo (o \textit{FTP} usa mais que uma conexão em simultâneo). Identifique, se aplicável, as fases de início de conexão, transferência de dados e fim de conexão. Identifique também os tipos de segmentos trocados e os números de sequência usados tanto nos dados como nas confirmações.
    
    \textbf{Resposta}: A transferência do file1 utilizando o protocolo FTP exige um número relativamente baixo de trocas de pacotes, como se conclui através dos diagramas.
    
    Na Figura 3, correspondente à transferência Servidor1-Portátil1, vê-se representada a fase de conexão nos primeiros picos. Observa-se também que a transferência do ficheiro exige mais trocas de pacotes, resultando no pico mais alto do diagrama. O último pico representa o fim da conexão ("\textit{Goodbye.}"). Durante todo o processo de comexão e transferência são trocados segmentos TCP e FTP. 
    
    
    \begin{figure}[!ht]
    \centering
    \includegraphics[scale=0.35]{img/a2 p1.png}
    \caption{Gráfico do \textit{Wireshark} utilizando TCP (segmentos a vermelhos) e FTP (segmentos a verde) no Portátil 1} 
    \label{fig:data20}
    \end{figure}
    
    \begin{figure}[!ht]
    \centering
    \includegraphics[scale=0.35]{img/a2 p1w.png}
    \caption{\textit{Wireshark} Portátil 1} 
    \label{fig:data21}
    \end{figure}
    
    \begin{figure}[!ht]
    \centering
    \includegraphics[scale=0.35]{img/a2 grilo.png}
    \caption{Gráfico do \textit{Wireshark} utilizando TCP (segmentos a vermelhos) e FTP (segmentos a verde) no Grilo} 
    \label{fig:data22}
    \end{figure}
    
    \begin{figure}[ht]
    \centering
    \includegraphics[scale=0.35]{img/a2 grilow.png}
    \caption{\textit{Wireshark} Grilo} 
    \label{fig:data23}
    \end{figure}
    
    \pagebreak
    \paragraph{}
    \paragraph{}
    \paragraph{}
    \paragraph{}
    \subsection{Pergunta 3} 
    Obtenha a partir do Wireshark, ou desenhe manualmente, um diagrama temporal para a transferência do ficheiro \textit{file1} por \textit{TFTP} realizada em A.4. Identifique, se aplicável, as fases de início de conexão, transferência de dados e fim de conexão. Identifique também os tipos de segmentos trocados e os números de sequência usados tanto nos dados como nas confirmações.
    
    
    \textbf{Resposta}: No caso da transferência por TFTP, não há uma fase de autenticação, simplesmente uma conexão ao servidor e pedido de transferência de ficheiro, pelo que o diagrama apresenta apenas um pico correspondente a estas ações.
    
    \textbf{NOTA:} na Figura 9 observam-se dois picos que correspondem ao procedimento experimental executado duas vezes (duas transferências do mesmo \textit{file1}).
    
    Durante o processo foram trocados segmentos do tipo: 
    \begin{itemize}
        \item Read Request
        \item Data
        \item Acknowledgement
    \end{itemize}
    
    \pagebreak
    
    \begin{figure}[!ht]
    \centering
    \includegraphics[scale=0.35]{img/a3 p1seg.png}
    \caption{Gráfico \textit{Wireshark} utilizando TFTP Portátil 1} 
    \label{fig:data33}
    \end{figure}
    
    \begin{figure}[!ht]
    \centering
    \includegraphics[scale=0.35]{img/a3 p1wir.png}
    \caption{\textit{Wireshark} Portátil 1} 
    \label{fig:data34}
    \end{figure}
    
    \begin{figure}[!ht]
    \centering
    \includegraphics[scale=0.35]{img/a3 griloseg.png}
    \caption{Gráfico \textit{Wireshark} utilizando TFTP Grilo} 
    \label{fig:data34}
    \end{figure}
    
    \begin{figure}[!ht]
    \centering
    \includegraphics[scale=0.35]{img/a3 grilowir.png}
    \caption{\textit{Wireshark} Grilo} 
    \label{fig:data34}
    \end{figure}
    
    \begin{figure}[!ht]
    \centering
    \includegraphics[scale=1 ]{img/a3 grilotimeout.png}
    \caption{\textit{Timeouts} Grilo} 
    \label{fig:data34}
    \end{figure}
    
    \pagebreak
    \paragraph{}
    \paragraph{}
    \paragraph{}
    \paragraph{}
    \paragraph{}
    \subsection{Pergunta 4}
    Compare sucintamente as quatro aplicações de transferência de ficheiros que usou, tendo em consideração os seguintes aspetos: (i) identificação da camada de transporte; (ii) eficiência; (iii) complexidade; (iv) segurança.
    
    \bigskip
    
    \textbf{Resposta}: 
    
    \begin{table}[]
    \resizebox{\columnwidth}{!}{%
    \begin{tabular}{|l|c|c|c|c|}
    \hline
    \multirow{2}{*}{Protocolos} & \multicolumn{1}{l|}{\multirow{2}{*}{\begin{tabular}[c]{@{}l@{}}Uso de camada \\   transporte (i)\end{tabular}}} & \multicolumn{1}{l|}{\multirow{2}{*}{\begin{tabular}[c]{@{}l@{}}Eficiência \\       (ii)\end{tabular}}} & \multicolumn{1}{l|}{\multirow{2}{*}{\begin{tabular}[c]{@{}l@{}}Complexidade \\         (iii)\end{tabular}}} & \multirow{2}{*}{\begin{tabular}[c]{@{}c@{}}Segurança\\  (iv)\end{tabular}}                                           \\
                                & \multicolumn{1}{l|}{}                                                                                           & \multicolumn{1}{l|}{}                                                                                  & \multicolumn{1}{l|}{}                                                                                       &                                                                                                                      \\ \hline
    SFTP                        & TCP                                                                                                             & Baixa                                                                                                  & Alta                                                                                                        & \begin{tabular}[c]{@{}c@{}}Alta\\ Encriptação de dados\end{tabular}                                                  \\ \hline
    FTP                         & TCP                                                                                                             & Médio                                                                                                  & Média                                                                                                       & \begin{tabular}[c]{@{}c@{}}Baixa\\ Autenticação, fácil descodificação de informação\end{tabular} \\ \hline
    TFTP                        & UDP                                                                                                             & Alta                                                                                                   & Baixo                                                                                                       & Baixa                                                                                                                \\ \hline
    HTTP                        & TCP                                                                                                             & Alta                                                                                                   & Baixa                                                                                                       & \begin{tabular}[c]{@{}c@{}}Muito baixa\\ Sem autenticação\end{tabular}                                               \\ \hline
    \end{tabular}%
    }
    \end{table}
    
    \paragraph{}
    O UDP provou ser mais rápido e eficiente uma vez que comparativamente ao TCP não reenvia pacotes previamente perdidos. Justificando alguns acontecimentos, entre eles a alta eficiência do protocolo TFTP em relação aos outros testados para este relatório.
    \paragraph{}
    \textbf{Nota}: Relativamente à baixa segurança do FTP, podemos verificar isso através de \ref{fig:data21}, onde conseguimos visualizar a password do usuário.
    
    \subsection{Pergunta 5} Com base no trabalho realizado, construa uma tabela informativa identificando, para cada aplicação executada (\textit{ping, traceroute, telnet, ftp, tftp, wget/lynx, nslookup, ssh, etc.}), qual o protocolo de aplicação, o protocolo de transporte, a porta de atendimento e o \textit{overhead} de transporte.
    
    \begin{center}
    % Please add the following required packages to your document preamble:
    % \usepackage{multirow}
    % \usepackage{graphicx}
    \begin{table}[]
    \resizebox{\columnwidth}{!}{%
    \begin{tabular}{|l|c|c|c|c|}
    \hline
    \multicolumn{1}{|c|}{\multirow{2}{*}{Aplicações}} & \multirow{2}{*}{\begin{tabular}[c]{@{}c@{}}Protocolo de \\  Aplicação\end{tabular}} & \multirow{2}{*}{\begin{tabular}[c]{@{}c@{}}Protocolo de \\ Transporte\end{tabular}} & \multirow{2}{*}{\begin{tabular}[c]{@{}c@{}}Porta de \\ Atendimento\end{tabular}} & \multirow{2}{*}{\begin{tabular}[c]{@{}c@{}}Overhead de Transporte\\ (em bytes)\end{tabular}} \\
    \multicolumn{1}{|c|}{}                            &                                                                                     &                                                                                     &                                                                                  &                                                                                              \\ \hline
    HTTP                                              & HTTP                                                                                & TCP                                                                                 & 80                                                                               & 20                                                                                           \\ \hline
    FTP                                               & FTP                                                                                 & TCP                                                                                 & 21                                                                               & 20                                                                                           \\ \hline
    TFTP                                              & TFTP                                                                                & UDP                                                                                 & 69                                                                               & 8                                                                                            \\ \hline
    Telnet                                            & Telnet                                                                              & TCP                                                                                 & 23                                                                               & 20                                                                                           \\ \hline
    Nslookup                                          & DNS                                                                                 & UDP                                                                                 & 53                                                                               & 8                                                                                            \\ \hline
    Ping                                              & ---                                                                                 & ---                                                                                 & ---                                                                              & ---                                                                                          \\ \hline
    \multicolumn{1}{|c|}{Traceroute}                  & DNS                                                                                 & UDP                                                                                 & 33434                                                                            & 8                                                                                            \\ \hline
    \end{tabular}%
    }
    \end{table}
    \end{center}
    
    \bigskip
    \pagebreak
    
\section{Conclusão}
    Com este estudo, tivemos a oportunidade de interiorizar os conceitos lecionados nas aulas teóricas fortificando as bases de Redes de Computadores e revisitamos os nossos conhecimentos associados ao uso de ferramentas como CORE e o \textit{Wireshark}.
    
    Passamos a ter uma noção um pouco mais aprofundada das as vantagens e desvantagens associadas à utilização dos diferentes protocolos TCP e UDP em contextos diferentes e aplicações de transferência de ficheiros entre \textit{hosts} como SFTP, FTP, TFTP e HTTP.
    
    O protocolo TCP provou ser menos eficiente, necessitando de mais recursos e sendo mais lento comparativamente ao UDP, estando associado aplicações onde se busca maior fiabilidade e segurança em troca de velocidade, uma vez que este garante a chegada do pacote esperando sempre a chegada de uma mensagem ACK antes do envio do pacote seguinte.
    
    Por outro lado, com UDP, é possível uma transferência mais rápida e eficiente abdicando da garantia de que o pacote chega ao destino, sendo assim um protocolo menos fiável. Geralmente é utilizado em serviços de streaming e VoIP onde a perca de um pacote não compromete totalmente a informação transmitida.
    

\end{document}
